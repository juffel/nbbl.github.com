%\documentclass[a4paper,twocolumn]{report}
\documentclass[a4paper,twocolumn]{scrartcl}
\addtokomafont{sectioning}{\rmfamily}
\usepackage[utf8]{inputenc}
\usepackage[ngerman]{babel}
\usepackage[left=1.5cm,right=1.5cm,top=2cm,bottom=2cm]{geometry}
\usepackage{enumitem}
\usepackage{amsmath}
\usepackage{amssymb}
\usepackage{color}
    \definecolor{lightgray}{gray}{.95}
\usepackage{listings}
    \lstset{
    %basicstyle=\usefont{T1}{pcr}{m}{n}\small, 
    %basicstlye=\small,
    numbers=none,
    numberstyle=\tiny,
    numbersep=5pt,
    backgroundcolor=\color{lightgray},
    showspaces=false,
    showstringspaces=false,
    showtabs=false, 
    frame=tb, 
    framerule=.8pt, 
    rulecolor=\color{black}, 
    tabsize=2,
    captionpos=b,
    breaklines=true,
    breakatwhitespace=true,
    title=\lstname,
    keywordstyle=\bfseries,
    stringstyle=\usefont{T1}{pcr}{m}{sl}\small,
    escapeinside={@}{@},
    morekeywords={process,in,out,is,begin,for,do,if,then, else, end, while}
    }
\newcommand{\lststyle}[1]{{\usefont{T1}{pcr}{m}{n}\small\selectfont#1}}

\author{Julian Dobmann, Frederik Feiten, Felix Herter, Johannes Sauer}
\title{Softwareprojekt Anwendung von Algorithmen}
\subtitle{Punkte in allgemeiner Lage}
 

\begin{document}

\twocolumn[
  \begin{@twocolumnfalse}
  \maketitle
  \vspace{-5ex}
  \rule{\textwidth}{1pt}
    \begin{abstract}
    \textbf{
      Bacon ipsum dolor sit amet meatloaf id ball tip anim sunt, corned beef ad pork loin prosciutto non velit ut nisi deserunt. Eu beef ex tri-tip meatball in pork chop, officia drumstick proident laborum. Aute short loin spare ribs corned beef. Proident labore kielbasa ham hock sint ut, fugiat nisi prosciutto chuck chicken. Pig fugiat bresaola, ut strip steak boudin proident officia. Elit andouille pork chop ut cow adipisicing short loin, in et laborum dolore in.}
    \end{abstract}
  \end{@twocolumnfalse}
]

\tableofcontents

\section{Einleitung}

\section{Datenstrukturen}

\subsection{Einführung der arithmetischen Unschärfe}
Bedingt durch die endliche Darstellung von Flie\ss kommazahlen im Prozessor waren wir gezwungen eine arithmetische
Unsch\"arfe einzuf\"uhren. \\
Wird zum Beispiel der Ausdruck $0.3-0.1$ berechnet, so wird anstelle des exakten Ergebnisses von $0.2$ 
als n\"achstbeste  Näherung:
\begin{lstlisting}
js> 0.3-0.1
0.19999999999999998
\end{lstlisting}
zur\"uckgeliefert. Die Differenz des exakten Wertes von der N\"aherung bel\"auft sich auf
\begin{lstlisting}
js> 0.2-(0.3-0.1)
2.7755575615628914e-17
\end{lstlisting}
, weicht also um die Gr\"o\ss enordung $10^{-16}$ von der, der Operanden ab.
Wir haben uns entschienden die arithmetische Unsch\"arfe auf $10^{-10}$ zu setzen. (\textbf{(remove me) Warum?}).
\subsection{Punkte und Vektoren}
Als grundlegende Datenstruktur wurden Vektoren eingef\"uhrt. Diese haben ihre zwei kartesischen Koordinaten als
Attribute und verf\"ugen \"uber die grundlegenden Eigenschaften und Funktionalit\"aten, welche von ihnen zu erwarten sind
(Vektoraddition/-subtraktion,
Multiplikation mit einem Skalar, Skalarmultiplikation, etc.). \textbf{(remove me) explizit dokumentieren}\\
Ohne in ihrer programmatischen Funktionalit\"at von den Vektoren abzuweichen, wurden Punkte eingef\"uhrt. Es stellte
sich heraus, dass beide Strukturen ihre Notwendigkeit hatten (der Abstand zweier Punkte sollte
als Vekor und 
nicht als Punkt ausgedr\"uckt werden, wohingegen ein Graph als Tupel von Verbindungskanten und Punkten, nicht Vektoren
dargestellt wird).
\textbf{(remove me) Was gibts noch zu sagen?}
\subsection{Kanten}
Die \lststyle{Edge} Datenstruktur repr\"asentiert eine  Strecke in einem kartesischen Koordinatensystem, als auch eine
Kante in einem Graph. Wir w\"ahlten aufgrund ihrer arithmetischen Vorz\"uge die Geradendarstellung in \emph{Hessescher
Normalform}. In ihr wird eine Gerade $g$ dargeschtellt durch ihren \emph{Normaleneinheitsvektor} $\widehat{n}$, sowie
durch ihren Abstand zum Koordinatenursprung. Jeder Punkt $p=(x,y)$ auf der Geraden erf\"ullt somit die Gelichung: 
\begin{align}
  \widehat{n}\vec{p}-d=0  
\end{align}
, wobei $\vec{p}$ der Ortsvektor von $p$ bezeichne.\\
Zus\"atzlich zu $\widehat{n}$ und $d$ speichert die \lststyle{Edge} Datenstruktur noch ihre zwei Endpunkte als
Attribute.
\subsubsection{Funktionen}
\begin{description}[font=\normalfont]
  \item{\lststyle{reload()}} 
    Aktualisiert f\"ur eine Kante ihren Normaleneinheitsvektor, sowie ihren Abstand zum Koordinatenursprung. Wird z.B.
    nach Verschieben eines Endpunktes aufgerufen.
  \item{\lststyle{length()}}
    Liefert die L\"ange einer Kante zur\"uck.
  \item{\lststyle{getY(x)}}
    Lifert die  $y$-Koordinate einer Geraden an $x$-Koordinate \lststyle{x}. \textbf{in fancy?}
  \item{\lststyle{getLeft()}}
    Liefert den Endpunkt einer Kante mit geringerer $x$-Koordinate. (Haben beide Endpunkte die selbe $x$-Koordinate so
    liefert \lststyle{getLeft()} einen anderen Punkt zur\"uck als \lststyle{getRight()}).  
  \item{\lststyle{getRight()}}
    Analog zu oben.
  \item{\lststyle{projectionToLine(pt)}}
    Liefert die Projektion eines Punktes \lststyle{pt} auf die Geraden durch \textbf{Alle Kanten benennen!}
  \item{\lststyle{projectionToEdge(pt)}}
    Analog zu \lststyle{ProjectionToLine}, leifert jedoch \lststyle{null} sollte die Projektion nicht in \textbf{Alle
    Kanten benennen!} liegen.
  \item{\lststyle{distanceToLine(pt)}}
    Liefert den Abstand eines Punktes zu
  \item{\lststyle{signedDistanceToLine(pt)}}
  \item{\lststyle{lineContains(pt)}}
  \item{\lststyle{contains(pt)}}
  \item{\lststyle{lineIntersection(edge)}}
  \item{\lststyle{edgeIntersection(egde)}}
\end{description}
\end{document}
